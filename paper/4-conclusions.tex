\section{Conclusions}\label{sec:conclusion}
    In this paper, we introduced a new method for running depletion simulations
    in OpenMC. This method relies on pre-calclated multi-group cross sections
    and fluxes stored in a \verb,.csv, file. The new method is siginifcantly
    faster than running a transport-coupled depletion simulation, albiet with a
    penalty to accuracy. Better accuracy will be obtained in simulations where
    the neutron flux spectrum will be constant, i.e. in fusion simulations.
         
    This study focused exclusively on nuclide compositions. No capability exists
    in \verb.IndependentOperator. to calculate $k_\text{eff}$ using. A good
    starting point may be in implementing a method similar to the method to estimate
    $k_{\infty}$ used in \cite{LOVECKY2014333}, wherein
    \begin{equation}
        k_{\infty} = \frac{\sum_{i} \sum_{j} N_{i} \nu_{j}
        \sigma_{i}^{j}}{\sum_{i}\sum_{j} N_{i} \sigma_{i}^{j}}  
    \end{equation}
    where $j$ indicates the type of reaction, and $\nu_{j}$ is the neutron
    multipliation of reaction $j$, defined as
    \begin{equation}
        \nu_j = \begin{cases}
            0 & j=\text{ absorption reaction}\\
            x & j=(n,xn)\\
            \bar{\nu} & j=\text{fission}
        \end{cases}
    \end{equation}
    In OpenMC, $(n,xn)$ reactions are not used to calculate $k_\text{eff}$ (and
    by extension $k_{\infty}$). In this case, $\nu_{x} = 0$ in the numerator for
    all $x$, and we would use an `effective' aborption cross section
    $\overline{\sigma}_{a} = \sigma_{a} - \sigma_{(n,2n)} - 2\sigma_{(n,3n)} -
    3\sigma_{(n,4n)}$. 

    The geometry used in this study was a simple PWR pincell. We did not find
    significant difference in error between one-group transport-independent
    depletion and 40-group transport-independent depletion. It is possible that 
    this distinction becomes more important for complex geometries.
