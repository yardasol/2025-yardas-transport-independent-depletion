\section{Conclusions}\label{sec:conclusion}
    In this paper, we introduced transport-independent depletion in OpenMC. This
    method utilized pre-computed multi-group cross sections and fluxes to
    calculate reaction rates for a depletion calculation. The new method is much
    faster than running a transport-coupled depletion simulation, albeit with a
    penalty to accuracy. Better accuracy will be obtained for models where the
    neutron flux spectrum will be constant, i.e. in fusion systems and low power
    fission reactors.
         
    This study focused exclusively on nuclide compositions. No capability exists
    in \verb.IndependentOperator. to monitor criticality as the fuel depletes. A
    good starting point may be in implementing a method similar to the method to
    estimate $k_{\infty}$ used in \cite{LOVECKY2014333}, wherein
    \begin{equation}
        k_{\infty} = \frac{\sum_{i} \sum_{j} N_{i} \nu_{j}
        \sigma_{i}^{j}}{\sum_{i}\sum_{j} N_{i} \sigma_{i}^{j}}  
    \end{equation}
    where $j$ indicates the type of reaction, and $\nu_{j}$ is the neutron
    multiplication of reaction $j$, defined as
    \begin{equation}
        \nu_j = \begin{cases}
            0 & j=\text{ absorption reaction}\\
            x & j=(n,xn)\\
            \bar{\nu} & j=\text{fission}
        \end{cases}
    \end{equation}
    In OpenMC, $(n,xn)$ reactions are not used to calculate $k_\text{eff}$ (and
    by extension $k_{\infty}$). In this case, $\nu_{x} = 0$ in the numerator for
    all $x$, and we would use an `effective' absorption cross section
    $\overline{\sigma}_{a} = \sigma_{a} - \sigma_{(n,2n)} - 2\sigma_{(n,3n)} -
    3\sigma_{(n,4n)}$. 

    Future work could focus on characterizing the effect of subregions on the
    concentration errors. The geometry used in this study is a simple PWR
    pincell. It is possible that the error due to static fluxes could be reduced
    by dividing the pincell into subregions to better capture radial and axial
    variations as is done in high-fidelity depletion of large reactor models.
    Another area for future work could be on studying the importance of energy
    group discretization for more complex reactor models. We did not find a
    difference in concentration errors between one-group, CASMO-8 group, and
    CASMO-40 group transport-independent depletion for the pincell model. It is
    possible that energy group discretization becomes more important for complex
    models with distinct depletion zones.    
