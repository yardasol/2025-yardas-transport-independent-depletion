\section{Results}\label{sec:results}
    To validate this new feature, we performed a parameter scan study on a PWR
    pincell using a single depletion zone. Table \ref{tab:mat-params} contains
    the material parameters of our model, and Table \ref{tab:geo-params}
    contains the geometric parameters.  We used the ENDF/B-VII.1 nuclear data
    library available at \url{openmc.org/official-data-libraries}. We used the
    ENDF/B-VII.1 depletion chain in the PWR spectrum available at
    \url{openmc.org/depletion-chains}.
    
    \begin{table}[<options>]
        \caption{Material Parameters}
        \label{tab:mat-params}
        \begin{tabular*}{\tblwidth}{@{}LLLL@{}}
            \toprule
             Item & Fuel & Cladding & Water \\ % Table header row
            \midrule
             Density [g/cc] & 10.4 & 6 & 1.0\\
             Volume [cc] & 0.1764$\pi$ & & \\
             S($\alpha$,$\beta$) &  & & \verb.c_H_in_H2O.\\
            \bottomrule
        \end{tabular*}
    \end{table}
    
    \begin{table}[<options>]
        \caption{Geometric Parameters}\label{tab:geo-params}
        \begin{tabular*}{\tblwidth}{@{}LLLL@{}}
            \toprule
            Fuel Radius & Clad Radius & Water Bounding Box dimensions\\
            \midrule
            0.42 & 0.45 &  1.24 $\times$ 1.24\\
            \bottomrule
        \end{tabular*}
    \end{table}
    We used timestep size and depletion timestepper as our parameters to vary.
    Table \ref{tab:timestep-shorthands} shows the different timestep sizes we
    used and their shorthand terminology. All simulations ran for 10 depletion
    steps.
    \begin{table}[<options>]
        \caption{}\label{tab:timestep-shorthands}
        \begin{tabular*}{\tblwidth}{@{}LL@{}}
            \toprule
            Timestep sizes & Shorthand term \\ % Table header row
            \midrule
            360 seconds & Minutes\\
            4 hours & Hours\\
            3 days & Days\\
            30 days & Months\\
            \bottomrule
        \end{tabular*}
    \end{table}
    We used two different depletion timesteppers: \verb.PredictorIntegrator.,
    analagous to Gauss' method, and \verb.CECMIntegrator., a predictor-corrector
    extension of Gauss' method.  We used \verb.fission-q. normalization for all
    cases with a linear power density of 174 W/cm.  For each parameter
    combination, we ran with three cases:
    \begin{enumerate}
        \item Transport-coupled depletion
        \item Transport-independent depletion
        \item Transport-independent depletion with microscopic cross sections
            updated after each depletion step (one-group only).
    \end{enumerate}

    For the first case, we used 25 inactive and 125 active batches, with 1e6
    particles per batch.  For transport independent depletion, we used three
    different group strucutres to test the effect on the accuracy: one group,
    CASMO-8, and CASMO-40.

    It should be noted that the third case is significantly slower than using
    transport-coupled depletion as the cross sections need to be reloaded after
    each depletion step. The third case was only run for one-group
    transport-independent depletion. To avoid a prohibitvely long paper, we will
    only show a subset of the results for each parameter. Unless otherwise
    specified, all results are using 3 day timesteps, and
    \verb.PredictorIntegrator.  The general trend is that errors are smaller for
    shorter depletion timesteps, and larger for longer depletion timesteps.
    Interested readers can find the full results -- including figure geneartion
    scripts for all cases -- at 10.5281/zenodo.15103043.

    %TODO: PUBLISH THE EXTERNAL DATA REPO

    % actinides error 

    \begin{figure}[h!tpb]
        \centering
        \includegraphics[width=\linewidth]{figs/actinides_constant_xs_predictor_fission_q_days.pdf}
        \caption[]{Actinide error in the fuel using constant cross sections.}
        \label{fig:actinides-error-constant-xs}
    \end{figure}
    Figures \ref{fig:actinides-error-constant-xs} and
    \ref{fig:actinides-error-updating-xs} show the relative error for actinides
    using constant cross sections and updating cross sections, respectively.  As
    expected, updating the cross sections at each depletion step results in very
    low errors, on the order of a fraction of a percent.  Using constant cross
    sections, the errors may be very low (5\% or less) or very high (more than
    10\%) depending on the nuclide of interest.
    \begin{figure}[htpb]
        \centering
        \includegraphics[width=\linewidth]{figs/actinides_updating_xs_predictor_fission_q_days.pdf}
        \caption[]{Actinide error in the fuel using updating cross sections}
        \label{fig:actinides-error-updating-xs}
    \end{figure}
    We suspect that the errors for isotopes of \ce{Am} and \ce{Cm} are so high
    are due to their extremely low composition in the fuel.
    %expand on this?
    % fission products error
    \begin{figure}[h!tpb]
        \centering
        \includegraphics[width=\linewidth]{figs/fission_products_constant_xs_predictor_fission_q_days.pdf}
        \caption{Fission product error using constant cross sections}
        \label{fig:fp-error-constant-xs}
    \end{figure}

    Figures \ref{fig:fp-error-constant-xs} and \ref{fig:fp-error-updating-xs}
    show the relative error for fission products using constant cross sections
    and updating cross sections, respectively.
    Similar to the actinides, updating the cross sections at each depletion step
    results in errors in the nuclide compositions  that are fractions of
    percents. However, unlike the actinides, the errors when using constant
    cross sections are all relavtively low, with the largest error being under
    3\%. A majority of our fissile material is uranium, specifically
    \ce{^{235}U}, and the error for that nuclide is so low that it doesn't even
    show on Figure \ref{fig:actinides-error-constant-xs}. It follows that the
    majority of our fission products come from \ce{^{235}U} in a composition
    that is equally low in error. The other actinides are present in amounts
    with higher error, but in very low quantities, so their effect on the error
    of the fission product compoistions is proportionaly small.
    
    \begin{figure}[htpb]
        \centering
        \includegraphics[width=\linewidth]{figs/fission_products_updating_xs_predictor_fission_q_days.pdf}
        \caption{Fission product error using  updating cross sections}
        \label{fig:fp-error-updating-xs}
    \end{figure}

    \begin{figure}[htpb]
        \centering
        \includegraphics[width=\linewidth]{figs/actinides_constant_xs_cecm_fission_q_days.pdf}
        \caption{Actinide error using the CECM integrator}
        \label{fig:actinides-error-cecm}
    \end{figure}


    % actinides predictor vs cecm
    Figure \ref{fig:actinides-error-cecm} shows the error for actinides using the
    \verb.CECM. integrator. Comparing these results with those of Figure
    \ref{fig:actinides-error-constant-xs} we do not see a significant difference
    in the error. For certain nuclides, the \verb.CECM. integrator appears to
    perform worse. We see a similar pattern for the fission products.

    \begin{figure}[htpb]
        \centering
        \includegraphics[width=\linewidth]{figs/actinides_constant_xs_predictor_source_rate_days.pdf}
        \caption{Actinide error using explicit value for the flux}
        \label{fig:actinides-error-source}
    \end{figure}

    Figure \ref{fig:actinides-error-source} shows the error for actinides using
    the flux we calculated from the initial material composition. These results
    do not differ significantly from the results in Figure
    \ref{fig:actinides-error-constant-xs}. The same is true for the fission
    products.

    Repeating this analysis for both the CASMO-8 and CASMO-40 multi-group
    strucutres did not yield significant decereases in error for
    transport-independent depletion over the one-group case. It is possible in
    more complication models, like full reactor depletion, that the multi-group
    strucutre will become more important..

    The time savings when using \verb.IndependentOperator. are immense. Whereas
    the transport-coupled simulations each took several hours to complete, the
    transport-independent simulations took seconds to minutes to complete!


