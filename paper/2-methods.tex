\section{Methodology}
    \label{sec:methods}
    Depletion is typically done in subregions that are small enough to have
    little spatial variation in the flux (and hence in the reaction rates). For
    example, in a full-core model, depletion regions may correspond to
    individual pincells, or even smaller sub-pin regions of radial rings and
    azimuthal segments. The general form of the equation governing depletion is
    %% The notation here matches what is in the OpenMC depletion paper from 2021
    \begin{equation}
      \label{eq:depletion}
      \begin{split}
        \frac{dN_i}{dt} = &\sum\limits_{j=1}^n \left[ \int_0^\infty 
        f_{j \rightarrow i}(E) \sigma_j (E, t) \phi(E,t) dE \;+ \lambda_{j\rightarrow i}
        \right] N_j(t) \\ &- \left [\int_0^\infty \sigma_i (E,t) \phi(E,t) dE \; +
        \sum\limits_{j=1}^n \lambda_{i\rightarrow j} \right ] N_i(t),
      \end{split}
    \end{equation}
    where
    \begin{equation*}
      \begin{split}
        N_i(t) &\equiv \text{density of nuclide $i$ at time $t$} \\
        \sigma_i(E,t) &\equiv \text{transmutation cross section for nuclide $i$ at energy $E$ and time $t$} \\
        \phi(E,t) &\equiv \text{neutron flux at energy $E$ and time $t$} \\
        f_{j \rightarrow i}(E) &\equiv \text{fraction of transmutation reactions in nuclide $j$ that produce nuclide $i$} \\
        \lambda_{j \rightarrow i} &\equiv \text{decay constant for decay modes in nuclide $j$ that produce nuclide $i$} \\
        n &\equiv \text{total number of nuclides.}
      \end{split}
    \end{equation*}
    The summations in Equation \ref{eq:depletion} are effectively over all
    nuclides except $i$, as $f_{i \rightarrow i}(E)$ and $\lambda_{i \rightarrow i}$ must
    be zero by definition. Equation \ref{eq:depletion} can be condensed into a
    matrix-vector equation:
    \begin{equation}
      \label{eq:depletion-matrix-t}
      \frac{d\vect{n}}{dt} = \vect{A}(\vect{n},t)\vect{n}
    \end{equation}
    where
    \begin{equation}
      \vect{n} = \begin{pmatrix} N_1 \\ N_2 \\ \vdots \\ N_n \end{pmatrix}, \quad
    \end{equation}
    and $\vect{A}(\vect{n},t) \in \mathbb{R}^{n\times n}$ is the depletion
    matrix containing the decay and transmutation coefficients. The coefficients
    of $\vect{A}$ change with time and depend on the nuclide
    composition vector $\vect{n}$, making this a nonlinear system of equations.
    OpenMC utilized the \Gls{cram} method to solve this system of equations
    \citep{romano_depletion_2021}. 
    
    We had two design goals for
    transport-independent depletion functionality:
    \begin{itemize}
        \item enable use of OpenMC depletion capabilities using only the Python API
        \item enable depletion of materials directly given a multi-group flux
            for each material.
    \end{itemize}
    
    We created the \verb.IndependentOperator. and \verb.MicroXS. classes to
    achieve these design goals.    
    \subsection{MicroXS}
        \label{sub:microxs}
        The \verb.MicroXS. class contains cross section data indexed by nuclide,
        reaction name, and energy group. It also contains methods to import and
        export \verb.MicroXS. objects from \verb,.csv, files. In principle, a
        user could create multi-group microscopic cross sections with a
        transport solver code, and then use the \verb.MicroXS.  class to read in
        the cross section data for use in depletion. However, this is
        unnecessary as the \verb.get_microxs_and_flux(). function runs an OpenMC
        $k$-eigenvalue calculation to create a \verb.MicroXS. object and
        multi-group flux profile for each user-specified domain.

        In the initial release of this feature in OpenMC v0.13.1, \verb.MicroXS.
        subclassed the Pandas \verb.DataFrame. class to store data and assumed a
        one-group structure. The v0.14.0 release removed the Pandas dependency
        and refactored \verb.MicroXS. class to store multi-group cross section
        data.

    \subsection{IndependentOperator}
        The \verb.Operator. class contains data and methods for running
        transport-coupled depletion calculations and runtime
        processing of OpenMC statepoint files to extract the data needed to
        solve Equation \ref{eq:depletion-matrix-t}. In this work, the
        \verb.Operator. class was refactored into a \verb.CoupledOperator.
        class maintaining the existing depletion capability, and the
        transport-independent depletion machinery was implemented into a new
        class, \verb.IndependentOperator. which uses an instance of the
        \verb.MicroXS. class and corresponding fluxes for each depletion domain
        to calculate reaction rates for that domain using multi-group microscopic
        cross sections.

        Calculating reactions rates using multi-group cross sections is
        mathematically equivalent to tallying the reaction rates directly. In
        transport-coupled depletion calculation, the transmutation reaction
        rates $r_{ij}(t)$ are tallied directly as:
        \begin{equation}
            \label{eq:cont-rxn-rate}
            r_{ij}(t) = \left[\int_0^\infty \sigma_{ij}(E,t) \phi(E,t) dE \; \right]
            N_{i}(t)
        \end{equation}
        where $i$ and $j$ indicate the nuclide and reaction, respectively.
            To perform depletion using multi-group cross sections, the multi-group
        cross sections and fluxes are multiplied to get a per-source neutron
        reaction rate:
        \begin{equation}
            \label{eq:mg-rxn-rate}
            r_{ij}(t) = \left[\sum_{g} \sigma_{ij,g}(t) \phi_{g}(t) \right]
            N_{i}(t) 
        \end{equation}
        In the limit of infinitely small energy groups, Equation
        \ref{eq:mg-rxn-rate} is equivalent to \ref{eq:cont-rxn-rate}. In
        practice, to get a time-dependent microscopic cross section or flux, we
        must perform transport calculations (and in the case of time-varying
        microscopic cross sections, incorporate thermal feedbacks). Herein lies
        the primary assumption of transport-independent depletion: {\it
        microscopic cross sections and reactor fluxes are static}.

        %% THIS DOESN'T MATCH WITH OPENMCs COMMENTS
        Both the tallied and multi-group-calculated reaction rates have units of
        $\text{reactions}/\text{src particle}$. A normalization factor
        $f$ in $\text{src}/\text{s}$ is applied to obtain the more
        conventional unit of $\text{reactions}/\text{s}$:
        \begin{equation}
            R_{ij} = r_{ij} f
        \end{equation}
        This normalization factor is typcially related to the power of the
        reactor and energy released from fission (\verb.fission-q.
        normalization), or the external source strength (\verb.source-rate.
        normalization). \verb.IndependentOperator. may use either of these methods to normalize
        the reaction rates. While \verb.CoupledOperator. may use several methods to
        calculate fission yields, including using the Monte Carlo simulation to
        directly tally fission yields, \verb.IndependentOperator. relies on
        constant yield data included in the depletion chain file.
