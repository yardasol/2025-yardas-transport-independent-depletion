\section{Methodology}
    \label{sec:methods}
    We had several design goals for transport-independent depletion functionality:
    \begin{itemize}
        \item Enable use of OpenMC depletion capabilities using only the Python API.
        \item Enable geometry-free depletion
        \item Allow users to pass one or more materials, and specify a power or
            flux along with a set of one-group microscopic cross sections to
            perform depletion.
    \end{itemize}
    
    We created the \verb.IndependentOperator. and \verb.MicroXS. classes to
    achieve these design goals.
    
    \subsection{MicroXS}
        The \verb.MicroXS. class is a subclass of the \verb,pandas.DataFrame,
        indexed by nuclide and reaction name. It also contains methods to create
        a \verb.MicroXS. object by reading a .csv file or a data array. In
        principle, a user could create one-group microscopic cross sections with
        a transport solver code, and then use the \verb.MicroXS. class to read in the
        cross section data for use in depletion. It should be noted that
        \verb.MicroXS. has a function to create one-group microscopic cross
        sections from an OpenMC \verb.Model. object that uses OpenMC's
        \verb.mgxs. module to process tallies.

        We chose one-group cross sections because they are easy to store and
        small, whereas a full continuous-energy cross section library can be on
        the order of gigabytes in size.
        
        Mathematically, calculating reactions rates using one-group cross
        sections should be equivalent to tallying the reaction rates directly.
        In a transport-coupled depletion calculation, reaction rates are tallied
        directly:
        % verify this, I have a suspicion we may need density and volume in there as well
        \begin{equation}
            r^j_i = \int_0^\infty \phi(E) \sigma^j_i(E) dE
        \end{equation}
        where $i$ and $j$ indicate the nuclide and reaction, respectively.

        These reaction rates are in units of reactions/src particle, so we need
        a normalization factor $f$ in src/s to get reactions/s:
        \begin{equation}
            R^j_i = r^j_i * f
        \end{equation}

        In obtaining one-group cross sections for a transport-independent
        depletion, we would also compute $r^j_i$ from a transport calculaton, as
        well as the energy-averaged flux (in units of particle-cm / src):
        \begin{equation}
            \bar{\phi} = \int_0^\infty \phi(E) dE
        \end{equation}
        and then compute the one-group macroscopic cross section:
        \begin{equation}
            \bar{\Sigma}^j_i = \frac{r^j_i}{\bar{\phi}}
        \end{equation}
        Notice that the src unit cancels out, leaving us with $cm^{-1}$, which
        are indeed the correct units.

        When we go to compute reaction rates, we use the equation 

        \begin{equation}
            R^j_i = \bar{\phi'} \bar\Sigma^j_i = r^j_i\frac{\bar{\phi'}}{\bar{\phi}} = r^j_i f
        \end{equation}
        where $\bar{\phi'}$ is the `true' neutron flux. By inspection, we can
        see that $f$ has the correct units of src/s. It follows that using a
        single energy group and energy-averaged flux to calculate reaction rates
        is mathematically equivalent to the statistical tallying of reaction
        rates.

            
    \subsection{IndependentOperator}
        The \verb.IndependentOperator. class uses an instance of the
        \verb.MicroXS. class to calculate the reaction rates using one-group
        microscopic cross sections via the following equation:
        \begin{equation}
            R^j_i = \bar{\phi'} \bar{\sigma}^j_i n_i V
        \end{equation}
        where $\sigma$ is the microscopic cross section, $n$ is the number
        density [$at/cm^3$] and $V$ is the volume [$cm^3$]. The microscopic
        cross sections are stored in a \verb.MicroXS. object which must be
        provided the user. This opens up indirect coupling to any transport code
        that can calculate one-group microscopic cross-sections. After some
        slight modifications to the existing OpenMC API, only these two classes
        are needed to perform transport independent depletion.

