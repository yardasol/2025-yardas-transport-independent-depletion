\section{Methodology}
    \label{sec:methods}
    We had two design goals for transport-independent depletion functionality:
    \begin{itemize}
        \item Enable use of OpenMC depletion capabilities using only the Python API.
        \item Enable depletion of materials directly given a multi-group flux
            for each material 
    \end{itemize}
    
    We created the \verb.IndependentOperator. and \verb.MicroXS. classes to
    achieve these design goals.    

    \subsection{MicroXS}
        \label{sub:microxs}
        The \verb.MicroXS. class contains cross-section data indexed by nuclide,
        reaction name, and energy group. It also contains methods to import and
        export \verb.MicroXS. objects from \verb,.csv, files. In principle, a
        user could create multi-group microscopic cross sections with a
        transport solver code, and then use the \verb.MicroXS.  class to read in
        the cross section data for use in depletion, however this is unnecessary
        as the \verb.get_microxs_and_flux(). function allows easy creation
        multiple \verb.MicroXS. objects and multi-group flux profiles for the
        user-specified domains.

        In the initial release of this feature in OpenMC v0.13.1, \verb.MicroXS.
        subclassed the Pandas \verb.DataFrame. class to store data and assumed a
        one-group structure. The v0.14.0 release removed the Pandas dependency
        and refactored \verb.MicroXS. class to store multi-group cross section
        data.
        
        Calculating reactions rates using multi-group cross sections is
        mathematically equivalent to tallying the reaction rates directly. In
        transport-coupled depletion calculation, reaction rates are tallied
        directly as:
        % verify this, I have a suspicion we may need density and volume in there as well
        \begin{equation}
            \label{eq:cont-rxn-rate}
            r^j_{i} = \int_0^\infty \phi(E) \sigma^j_i(E) dE
        \end{equation}
        where $i$ and $j$ indicate the nuclide and reaction, respectively.
            To perform depletion using multi-group cross sections, the multi group
        cross sections and fluxes are multiplied to get a per-source neutron
        reaction rate:
        \begin{equation}
            \label{eq:mg-rxn-rate}
            r^j_{i} = \sum_{g} \phi_{g} \sigma^{j}_{i,g} 
        \end{equation}
        calculation. In the limit of infinitely small energy groups,  OpenMC

        %% THIS DOESN'T MATCH WITH OPENMCs COMMENTS
        Both the tallied and multi-group-calculated reaction rates have units of
        $\frac{\text{reactions}}{\text{src particle}}$
        A normalization factor $f$ in $\frac{\text{src}}{\text{s}}$ is applied
        to obtain the more conventional unit of
        $\frac{\text{reactions}}{\text{s}}$:
        \begin{equation}
            R^j_i = r^j_i * f
        \end{equation}
        This normalization factor is typcially related to the power of the
        reactor or external source strength.

    \subsection{IndependentOperator}
        The \verb.Operator. class was refactored into a \verb.CoupledOperator.
        class maintaining the existing depletion cabability, and the
        transport-independent depletion machinery was implemented into a new
        class, \verb.IndepednentOperator. which uses an instance of the
        \verb.MicroXS. class to calculate the reaction rates using one-group
        microscopic cross sections via the following equation:
        \begin{equation}
            R^j_i = \bar{\phi'} \bar{\sigma}^j_i n_i V
        \end{equation}
        where $\sigma$ is the microscopic cross section, $n$ is the number
        density [$at/cm^3$] and $V$ is the volume [$cm^3$]. The microscopic
        cross sections are stored in a \verb.MicroXS. object which must be
        provided the user. This opens up indirect coupling to any transport code
        that can calculate one-group microscopic cross-sections. After some
        slight modifications to the existing OpenMC API, only these two classes
        are needed to perform transport independent depletion.

        Unlike \verb.CoupledOperator., \verb.IndependentOperator. cannot use the
        full range of helper classes for calcultating fission yields,
        normalizing the tally results, and computing reaction rates. Reaction
        rates are computed as described in Section \ref{sub:microxs}, whereas
        fission yields are computed from yield data in the provided chain file.
        The reaction rates are normalized from
        $\frac{\text{reactions}}{\text{src}}\frac{\text{b-cm}}{\text{atom}}$ to
        $\frac{\text{reactions}}{\text{atom-s}}$ using the\ldots
        

        Multi-group fluxes should be normalized. They can then be freely scaled
        by any source rate.

