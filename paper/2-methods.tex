\section{Methodology}
    \label{sec:methods}
    We had two design goals for transport-independent depletion functionality:
    \begin{itemize}
        \item Enable use of OpenMC depletion capabilities using only the Python API.
        \item Enable depletion of materials directly given a multi-group flux
            for each material 
    \end{itemize}
    
    We created the \verb.IndependentOperator. and \verb.MicroXS. classes to
    achieve these design goals.    

    \subsection{MicroXS}
        The \verb.MicroXS. class is a subclass that contains cross-section data
        indexes by nuclide, reaction name, and energy group. It also contains
        methods to import and export \verb.MicroXS. objects from \verb,.csv,
        files. In principle, a user could create multi-group microscopic cross
        sections with a transport solver code, and then use the \verb.MicroXS.
        class to read in the cross section data for use in depletion, however
        this is unnecessary as the \verb.get_microxs_and_flux(). function allows
        easy creation multiple \verb.MicroXS. objects and multi-group flux
        profiles for the user-specified domain.

        In the initial release of this feature in OpenMC v0.13.1, \verb.MicroXS.
        subclassed the Pandas \verb.DataFrame. class to store data and assumed a
        one-group structure. The v0.14.0 release removed the Pandas dependency
        and refactored \verb.MicroXS. class to store multi-group cross section
        data.
        
        Calculating reactions rates using multi-group cross
        sections is mathematically equivalent to tallying the reaction rates
        directly. In transport-coupled depletion calculation, reaction rates
        are tallied directly as:
        % verify this, I have a suspicion we may need density and volume in there as well
        \begin{equation}
            r^j_{i} = \int_0^\infty \phi(E) \sigma^j_i(E) dE
        \end{equation}
        where $i$ and $j$ indicate the nuclide and reaction, respectively.

        These reaction rates are in units of reactions/src particle, so we need
        a normalization factor $f$ in src/s to get reactions/s:
        \begin{equation}
            R^j_i = r^j_i * f
        \end{equation}

        In obtaining multi-group cross sections for a transport-independent
        depletion, we would also compute the group-wise reaction rate
        $r^j_{i,g}$ from a transport calculaton, as well as the group-averaged
        flux (in units of particle-cm / src):
        \begin{equation}
            \bar{\phi}_{g} = \int_{E_{g}}^{E_{g-1}} \phi(E) dE
        \end{equation}
        and then compute the multi-group macroscopic cross section:
        \begin{equation}
            \bar{\Sigma}^j_{i,g} = \frac{r^j_{i,g}}{\bar{\phi}_{g}}
        \end{equation}
        Notice that the src unit cancels out, leaving us with $cm^{-1}$, which
        are indeed the correct units.

        When we go to compute reaction rates, we use the equation 

        \begin{equation}
            R^j_i = \sum_{g} \bar{\phi'}_{g} \bar\Sigma^j_{i,g} = \sum_{g}
            r^j_{i,g}\frac{\bar{\phi'}}{\bar{\phi}} = r^j_i f
        \end{equation}
        where $\bar{\phi'}$ is the `true' neutron flux. By inspection, we can
        see that $f$ has the correct units of src/s. It follows that using a
        single energy group and energy-averaged flux to calculate reaction rates
        is mathematically equivalent to the statistical tallying of reaction
        rates.

            
    \subsection{IndependentOperator}
        The \verb.IndependentOperator. class uses an instance of the
        \verb.MicroXS. class to calculate the reaction rates using one-group
        microscopic cross sections via the following equation:
        \begin{equation}
            R^j_i = \bar{\phi'} \bar{\sigma}^j_i n_i V
        \end{equation}
        where $\sigma$ is the microscopic cross section, $n$ is the number
        density [$at/cm^3$] and $V$ is the volume [$cm^3$]. The microscopic
        cross sections are stored in a \verb.MicroXS. object which must be
        provided the user. This opens up indirect coupling to any transport code
        that can calculate one-group microscopic cross-sections. After some
        slight modifications to the existing OpenMC API, only these two classes
        are needed to perform transport independent depletion.

        Multi-group fluxes should be normalized. They can then be freely scaled
        by any source rate.

