\section{Introduction}\label{}
    Modeling and simulation codes will play a critical role in licensing
    advanced nuclear reactors. In preparation for advanced nuclear, both the
    \Gls{doene} and the \Gls{nrc} identified several technical gaps in current
    \Gls{mns} tools that are necessary for efficient and effective license
    application reviews \cite{betzler_modeling_2019} \cite{usnrc_nonlwr_2020-1}.
    The composition of nuclides in a material subject to irradiation over time
    changes due to the nuclear transmutation and decay reactions taking place.
    This process is commonly known as {\it depletion} or {\it burnup}.  Both
    \Gls{doene} and the \Gls{nrc} have specifically identified depletion to be a
    key modeling feature for advanced nuclear reactors.
    
    Modeling depletion in nuclear reactors is inherently linked with modeling
    neutron transport. In order to obtain the transmutation reaction rates
    needed for a depletion calculation, one needs to calculate the neutron flux
    by solving the neutron transport equation. Conversely, after every depletion
    calculation, the transport solution will need  to be run again due the
    change in material composition. This process can be very expensive if
    looking at lifetime behavior of reactor designs, as many transport
    simulations are needed and they must be done sequentially.

    The ExaSMR (Small modular reactor) project seeks to use state-of-the-art
    software tools in Monte Carlo transport and compuitational fluid dynamics
    coupled with heat transfer for simulating advanced reactors along with
    exascale computing hardware to overcome this computational challenge. These
    methods are very computationally expensive, but deliver high confidence in
    their results. If one component in this tool chain could be simplified or
    sped up, it would greatly reduce the computational cost on and economic cost
    from these exascale machines.

    Depletion capabilities were recently added to OpenMC, an open source,
    community developed MC particle transport code \cite{romano_openmc_2015}
    \cite{romano_depletion_2021}. The depletion solver utilizes OpenMC's Python
    API to iteratively run a transport simulation, calculate the reaction rates,
    and update the material composition. In the present work, we describe a new
    method for running depletion calculations without the need to iteratively
    run transport calculations by using precalculated microscopic cross sections
    for each depletion step. At most, only one transport calculation needs to be
    run to generate these cross sections.

    The paper is organized as follows. In Section \ref{sec:methods}, we describe
    the methods and algorithms used to calculate reaction rates using the new
    depletion capabilities. In Section \ref{sec:results}, we describe our
    simulation and analyse the results. In Section \ref{sec:conclusion}, we
    summarize our results, and discuss some gaps in the current implementation
    of the new feature.


