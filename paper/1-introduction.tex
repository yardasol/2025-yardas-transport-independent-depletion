\section{Introduction}\label{}
    Modeling and simulation codes will play a critical role in licensing
    advanced nuclear reactors. In preparation for advanced nuclear, both the
    \Gls{doene} and the \Gls{nrc} identified several technical gaps in current
    \Gls{mns} tools that are necessary for efficient and effective license
    application reviews \citep{betzler_modeling_2019, usnrc_nonlwr_2020-1}.
    The composition of nuclides in a material subject to irradiation over time
    changes due to the nuclear transmutation and decay reactions taking place.
    This process is commonly known as {\it depletion} or {\it burnup}.  Both
    \Gls{doene} and the \Gls{nrc} have specifically identified depletion to be a
    key modeling feature for advanced nuclear reactors.

    Recent projects seek to use state-of-the-art software tools in Monte Carlo
    transport and computational fluid dynamics coupled with heat transfer for
    simulating advanced reactors along with exascale computing hardware to
    overcome this computational challenge~\citep{romano2021nse,merzari2023sc}.
    These methods are very computationally expensive, but deliver high
    confidence in their results. If one component in this tool chain could be
    simplified or sped up, it would greatly reduce the computational cost on and
    economic cost from these exascale machines.

    Depletion capabilities were recently added to OpenMC, an open source,
    community developed MC particle transport code \citep{romano_openmc_2015,
    romano_depletion_2021}. The depletion solver utilizes OpenMC's Python API to
    iteratively run a transport simulation, calculate the reaction rates, and
    update the material composition. In the present work, we describe a new
    method for running depletion calculations without the need to iteratively
    run transport calculations by using precalculated microscopic cross sections
    for each depletion step. At most, only one transport calculation needs to be
    run to generate these cross sections.

    In a reactor depletion calculation, there is a tight coupling between
    neutron transport and depletion because the change in material composition
    can result in changes to the spatial distribution of reaction rates. This
    limits the practical application of transport-independent depletion to
    domains where capturing spatial effects are a secondary concern. One
    such application is in global fuel-cycle analysis where we are more
    concerned with the overall change in material composition. Fuel cycle
    simulations typically model many different reactors simultaneously over
    long time periods. Performing transport-coupled depletions for each reactor
    at each timestep would be much more expensive than using
    transport-independent depletion. The capabilites described in this paper
    have been utilized in Cyclus \cite{huff_fundamental_2016}, an open source
    fuel cycle simulator, for coupling depletion to the fuel cycle
    simulaton \cite{bachmann_os_2024}. However, for other applications, the
    coupling between neutron transport and depletion is effectively one-way.
    This is often the case for experimental facilities with a neutron source
    where the source rate is high enough to cause activation of materials but
    not high enough to produce significant burnup of materials. Many fusion
    energy facilities fall into this category.  For such applications, there is
    no need to run a fully coupled simulation between neutron transport and
    depletion; instead, a neutron transport simulation corresponding to the
    beginning of irradiation can be carried out to determine reaction rates
    relative to the neutron source rate, and then the depletion calculation can
    be performed using the reaction rates scaled by any changes in the neutron
    source rate over time. For fusion systems, these calculations are often
    carried out with FISPACT-II~\citep{sublet2017nds}; see \citet{eade2020nf}
    for a recent example. The capabilities described in this paper have also
    been utilized for a recent study of shutdown dose rate in fusion
    systems~\citep{peterson2024nf}.

    The paper is organized as follows. In Section \ref{sec:methods}, we describe
    the methods and algorithms used to calculate reaction rates using the new
    depletion capabilities. In Section \ref{sec:results}, we describe our
    simulation and analyse the results. In Section \ref{sec:conclusion}, we
    summarize our results, and discuss some gaps in the current implementation
    of the new feature.


