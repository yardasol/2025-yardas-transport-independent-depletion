\section{Introduction}\label{}
    %Modeling and simulation codes will play a critical role in licensing
    %advanced nuclear reactors. In preparation for advanced nuclear technology,
    %both the \Gls{doene} and the \Gls{nrc} identified several technical gaps in
    %current \Gls{mns} tools that are necessary for efficient and effective
    %license application reviews \citep{betzler_modeling_2019,
    %usnrc_nonlwr_2020-1}. Both \Gls{doene} and the \Gls{nrc} have specifically
    %identified depletion to be a key modeling feature for advanced nuclear
    %reactors.

    Accurate simulations of advanced reactors on exascale computing hardware
    requires state-of-the-art software tools in Monte Carlo transport and
    heat-coupled computational fluid dynamics
    ~\citep{romano2021nse,merzari2023sc}.  These methods are computationally
    expensive, but deliver high confidence in their results. If any component in
    this tool chain could be simplified or sped up, it would reduce the
    computational cost on and economic cost from these exascale machines.

    {\it Depletion} is the process by which the nuclide composition changes due
    to nuclear transmutation and decay reactions occurring in a material subject
    to irradiation. Depletion capabilities were recently added to OpenMC, an
    open source, community developed MC particle transport code
    \citep{romano_openmc_2015, romano_depletion_2021}. The depletion solver
    utilizes OpenMC's Python API to iteratively run a transport simulation,
    calculate the reaction rates, and update the material composition.
    In a reactor depletion calculation, there is a tight coupling between
    neutron transport and depletion because the change in material composition
    can result in changes to the spatial distribution of reaction rates.

    In the present work, we describe a new method for running depletion
    calculations without the need to iteratively run transport calculations by
    using pre-calculated microscopic cross sections and fluxes for each depletion
    step. At most, only one transport calculation needs to be run to generate
    these cross sections and fluxes. If few materials are used, this effectively
    removes spatial effects of the changing material composition due to the
    static fluxes and cross sections. It is possible that spatial effects could
    be approximated by defining several materials for each fuel pin to capture
    radial and spatial effects, but this has not been tested.

    This limits the practical application of transport-independent depletion to
    domains where capturing spatial effects are a secondary concern. One
    such application is in global fuel-cycle analysis where we are more
    concerned with the overall change in material composition. Fuel cycle
    simulations typically model many different reactors simultaneously over
    long time periods. The capabilities described in this paper
    have been utilized in Cyclus \citep{huff_fundamental_2016} -- an open source
    fuel cycle simulator -- for coupling depletion to the fuel cycle
    simulaton \citep{bachmann_os_2024}. However, for other applications, the
    coupling between neutron transport and depletion is effectively one-way.
    This is often the case for experimental facilities with a neutron source
    where the source rate is high enough to cause activation of materials but
    not high enough to produce significant burnup of materials. Many proposed fusion
    energy facilities would fall into this category.  For such applications, there is
    no need to run a fully coupled simulation between neutron transport and
    depletion; instead, a neutron transport simulation corresponding to the
    beginning of irradiation can determine reaction rates relative to the
    neutron source rate, and then the depletion calculation can be performed
    using the reaction rates scaled by any changes in the neutron source rate
    over time. For fusion systems, these calculations are often carried out with
    FISPACT-II~\citep{sublet2017nds}; See \citet{eade2020nf} for a recent
    example. The capabilities described in this paper have also been utilized
    for a recent study of shutdown dose rate in fusion
    systems~\citep{peterson2024nf}.

    This paper is organized as follows. In Section \ref{sec:methods}, we describe
    the methods and algorithms used to calculate reaction rates using the new
    depletion capabilities. In Section \ref{sec:results}, we describe our
    simulation and analyze the results. In Section \ref{sec:conclusion}, we
    summarize our results, and discuss some gaps in the current implementation
    of the new feature.


